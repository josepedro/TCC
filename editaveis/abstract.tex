\begin{resumo}[Abstract]
 \begin{otherlanguage*}{english}
   Currently the song is a single level with regard to various approaches to behold and run and, therefore, the technology is increasingly becoming an interaction approach with the musical processes. One of the technology examples are automatic music transcription systems that help the musician, replacing sometimes significantly scores, tabs and chords. This present study aims to develop a computational solution for recognition of musical harmonies. For this purpose focused on the implementation of spectral analysis of the audio sample, classification of musical notes, chord classification with support inversion, recognition of rhythmic and harmonic transition patterns over time. The development of the solution took on a transdisciplinary perspective (complexity theory) with the help of scientific methodology, using Scilab programming language for implementation. Of theoretical foundations were used physical concepts of sound, music theory, signal processing and artificial neural networks. The development of the solution allowed the detection of chords in major triads, minor, augmented and diminished in an audio sample.

   \vspace{\onelineskip}

   \noindent
   \textbf{Key-words}: recognition. chords. music. processing. signals. networks. neural.
 \end{otherlanguage*}
\end{resumo}
