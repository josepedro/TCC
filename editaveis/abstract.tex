\begin{resumo}[Abstract]
 \begin{otherlanguage*}{english}
   Currently the music is a single with regard to various approaches to contemplate and execute and, with it, the technology is increasingly becoming an approach for interaction with the musical level processes. One of the examples of technology are automatic music transcription systems that help the musician, replacing sometimes significantly scores, tabs and chords. This present study aims to develop a computational solution for the recognition of musical harmonies. For this purpose used techniques of signal processing and artificial neural networks. Developing solution allowed the recognition of chords in major, minor, augmented and diminished triads.

   \vspace{\onelineskip}

   \noindent
   \textbf{Key-words}: recognition. chords. music. processing. signals. networks. neural.
 \end{otherlanguage*}
\end{resumo}
