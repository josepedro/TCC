\begin{resumo}
 Atualmente a música está num patamar único no que diz respeito a várias abordagens de se contemplar e se executar e, com isso, a tecnologia vem cada vez mais se tornando uma abordagem de interação com os processos musicais. Um dos exemplos de tecnologia são sistemas automáticos de transcrição de música que auxiliam o músico, substituindo por vezes de maneira significativa partituras, tablaturas e cifras. Esse presente trabalho tem como objetivo desenvolver uma solução computacional para reconhecimento de harmonias musicais. Para tal fim focou-se na implementação da análise espectral da amostra de áudio, classificação em notas musicais, classificação em acordes com suportes a inversões, transição rítmica e reconhecimento dos padrões harmônicos ao longo do tempo. O desenvolvimento da solução se deu sobre uma perspectiva transdisciplinar (teoria da complexidade) com o auxílio da metodologia científica, utilizando a linguagem de programação Scilab para implementação. De fundamentos teóricos foram utilizados conceitos físicos do som, teoria musical, processamento de sinais e redes neurais artificiais. O desenvolvimento da solução permitiu a detecção de acordes em tríades maiores, menores, aumentados e diminutos numa amostra de áudio.

 \vspace{\onelineskip}

 \noindent
 \textbf{Palavras-chaves}: reconhecimento. acordes. música. processamento. sinais. redes. neurais. harmonia.
\end{resumo}
