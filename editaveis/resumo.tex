\begin{resumo}
 Atualmente a música está num patamar único no que diz respeito a várias abordagens de se contemplar e se executar e, com isso, a tecnologia vem cada vez mais se tornando uma abordagem de interação com os processos musicais. Um dos exemplos de tecnologia são sistemas automáticos de transcrição de música que auxiliam o músico, substituindo por vezes de maneira significativa partituras, tablaturas e cifras. Esse presente trabalho tem como objetivo desenvolver uma solução computacional para detecção de acordes musicais. Para tal fim utilizou-se técnicas de processamento de sinais e redes neurais artificiais. O desenvolvimento da solução permitiu a detecção de acordes em tríades maiores, menores, aumentados e diminutos.

 \vspace{\onelineskip}
    
 \noindent
 \textbf{Palavras-chaves}: detector. acordes. música. processamento. sinais. redes. neurais.
\end{resumo}
