\chapter{Considerações Finais}
\label{chap:conclusoes}

Em vista do que foi exposto, conclui-se que o sistema é de viabilidade significativa no que tange a aplicação e função principal: reconhecimento de harmonias musicais.

O sistema reconhece com 100\% de eficácia os acordes de todas as possibilidades em tríades gravados em amostras separadas, ou seja, cada acorde, no instrumento musical piano, gravado separadamente em cada arquivo de áudio.

No que diz respeito ao reconhecimento de acordes ao longo do tempo, o sistema possui 75\% de eficácia para acordes de piano e 50\% para acordes de violão, ambos tocados ao longo do tempo. Músicas completas não foram testadas pois o presente trabalho focou somente no reconhecimento de acordes ao longo do tempo em contextos polifônicos e mono-instrumentais. Subtende-se que o resultado foi comprometido pelo fato do sistema não ser constituído de um módulo de detecção de acordes não só ao longo do tempo, mas, dado um conjunto de possibilidades harmônicas, resultados em que o acorde se comportaria como totalmente não harmônico poderia ser tratado como um ruído ou silência e não como uma sugestão de acorde. Vale ressaltar que o sistema não possui filtros para pré-processamento de áudio focando a redução de ruídos.

No que tange a extração de tons para as músicas, o sistema funcionou com 67\% de eficácia global, sendo testadas inclusive músicas polifônicas e multi-instrumentais. Devido a dependência dessa funcionalidade com o módulo de reconhecimento de acordes ao longo do tempo, o resultado poderia melhorar com aprimorações nesse módulo. 

\section{Evoluções Futuras}

Em relação a evoluções futuras, é passível de consideração atualização do sistema com as técnicas de estado da arte publicadas no ISMIR\footnote{http://ismir.net/}. Também pode-se considerar aprimorações no sistema de reconhecimento de notas musicais considerando tipos diferentes de instrumentos.

Sincronizar a extração de notas e acordes com reconhecimento de padrões rítmicos focando começo e fins de compassos também seria de grande importância para distinção de notas presentes em acordes. Também seria de grande valia para o sistema a implementação de um módulo de pré-processamento de sinal de áudio com o intuito de amenizar ruídos.


