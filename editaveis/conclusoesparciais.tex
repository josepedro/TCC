\chapter{Conclusões}
\label{chap:conclusoes}

Em vista do que foi exposto, conclui-se que o sistema é de viabilidade significativa no que tange a aplicação e função principal: reconhecimento de acordes num conjunto de amostras de sinal de áudio.

É importante ressaltar que na segunda camada, rede neural para sugestão de notas, há a presença de um certo erro com relação às frequências de suas determinadas notas. Deve-se portanto ajustar e calibrar com mais acurácia as faixas presentes em cada nota musical. Ajustando esse detalhe o sistema fica mais preciso para a sugestão de notas.

No que diz respeito a futuras evoluções, é passível de consideração o uso das transformadas wavelets para o aprimoramento da detecção de acordes localizados no tempo. É desejável um algorítmo para análise de audio na detecção de transições rítmicas ao longo do tempo, focando localizar aonde os acordes se encontram num determinado compasso musical. Também há a possibilidade de implementação do sistema num produto de software, mais especificamente numa plataforma móvel Android.